% ***** Kopf- und Fußzeilen *****
%\usepackage[automark]{scrpage2}
%\addtolength{\footnotesep}{2pt}		%% Abstand der Fussnote zur Trennlinie
%\setcounter{secnumdepth}{3}			%% Nummerierung der Kapitel
%\setcounter{figure}{4}		        	%% Bilder
%\setcounter{tocdepth}{3}			%% Gliederungsebene fuer Inhaltsverzeichnis
%\clearscrheadings                      	%% Entferne alle vorhandenen header
%\pagestyle{scrheadings}                	%% Nutze scrheader
%\addtokomafont{pagehead}{\normalsize\upshape}	%% nichtkursive Kopf-/Fusszeilen
%\setheadsepline{.05pt}				%% Trennlinie oben
%\setfootsepline{.05pt}				%% Trennlinie unten
%\ihead{\leftmark}
%\ofoot{\pagemark}
%\addtolength{\headheight}{15mm}
%\ohead[{\includegraphics[height=60pt]{MPIB2}}]{\includegraphics[height=60pt]{MPIB2}}

% ***** Seitenlayout *****
\usepackage[top=3.5cm,left=3.5cm,right=2.5cm,bottom=3.0cm]{geometry}
\setcounter{secnumdepth}{5}         		%% Durchnummerierung
\usepackage[onehalfspacing]{setspace}   	%% Anderthalbzeiliger Abstand
\usepackage{mathptmx}               		%% De­fines Adobe Times Ro­man (or equiv­a­lent) as de­fault text font, and pro­vides maths sup­port us­ing glyphs from the Sym­bol, Chancery and Com­puter Modern fonts to­gether with let­ters
\usepackage[scaled=.90]{helvet}
\usepackage{courier}

% ***** Wortabstaende *****
%\frenchspacing					%% keine laengeren Leerzeichen nach Satzende/Abkuerzungen mit Punkt
%\setlength{\parindent}{0pt}        		%% kein Einzug bei neuem Absatz
%\setlength{\parskip}{1.5ex plus0.5ex minus 0.5ex}	%% Abstand zwischen zwei Absaetzen
%\tolerance 1414                    		%% Influences the paragraph breaking routine itself
%\hbadness 1414                     		%% Influences the user report (the messages you see on screen and in the log) about the actually chosen lines
%\setlength{\emergencystretch}{1.5pt}
%\hfuzz 0.3pt                       		%% Parameter that allows hbox's to be overfull by [length] before an overfull error occurs
%\widowpenalty=10000                		%% Penalty for a broken page, with a single line of a paragraph (called "widow") remaining on the top of the succeeding page
%\vfuzz \hfuzz                      		%% Show hfuzz
%\raggedbottom                      		%% Makes all pages the height of the text on that page. No extra vertical space is added
%\brokenpenalty=10000               		%% Penalty for a page break, where the last line of the previous page contains a hyphenation

% ***** Packages *****
\usepackage{scrhack}				%% Hacks, um Pakete mit KOMA-Klassen kompatibel zu machen
%\usepackage{graphicx}				%% Grafiken
\usepackage{multicol}				%% Mehrspaltig
\usepackage{booktabs}				%% Qualitativ hochwertige horizontale Linien in Tabellen
%\usepackage{url}				%% URLs einbinden
%\usepackage{textcomp}				%% Verschiedene Symbole und Sonderzeichen
\usepackage{caption}[2008/08/24]		%% Unterschriften für Bilder, Tabellen, etc.
\usepackage{babel}				%% Paket für Silbentrennung etc.
\usepackage[T1]{fontenc}            		%% Schriftkodierung
\usepackage[utf8]{inputenc}			%% Eingabecodierung
%\usepackage{float}                  		%% Im­proves the in­ter­face for defin­ing float­ing ob­jects such as fig­ures and ta­bles
%    \restylefloat{figure}           		%% Restyle Figure
\usepackage{lmodern}				%% Schriftart
%\usepackage{wrapfig}                		%% Allows in-line images such as the example fish picture
%\usepackage{listings}				%% Erlaubt Syntax-Korrektes Code-Einbinden
\usepackage{color}                  		%% Farben freischalten
\usepackage[usenames,dvipsnames,svgnames,table]{xcolor}		%% Used for tables
%\graphicspath{{pictures/}}          		%% Specifies the directory where pictures are stored
\renewcommand*\familydefault{\sfdefault}	%% Umschaltung auf Sans Serif
\usepackage[pdfborder={0 0 0}, colorlinks=false]{hyperref}	%% Klickbare Verweise
%\usepackage{nomencl}                		%% Abbkuerzungsverzeichnis
%\makenomenclature                   		%% Abbkuerzungsverzeichnis aufbauen

% ***** Bibliographie-Setup *****
%\usepackage[style=footnote-dw, backend=bibtex8]{biblatex} %% Geisteswissenschaften, speziell für Fußnoten
%\usepackage[style=authoryear-icomp, backend=bibtex8]{biblatex} %% Geisteswissenschaften, Autor-Jahr
%\usepackage[style=alphabetic-verb, backend=bibtex8]{biblatex} %% Naturwissenschaften
%\usepackage[babel,german=quotes]{csquotes}
%\bibliography{bibliography/literature.bib}

% ***** Funktionen *****
\newcommand{\changefont}[3]{
    \fontfamily{#1} \fontseries{#2} \fontshape{#3} \selectfont}
